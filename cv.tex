\documentclass{resume}
\usepackage{zh_CN-Adobefonts_external} 
\usepackage{linespacing_fix}
\usepackage{cite}
\usepackage{hyperref}
\hypersetup{
    colorlinks=true,
    linkcolor=cyan,
    filecolor=magenta,      
    urlcolor=blue,
}
\usepackage{fancyhdr}

\renewcommand{\refname}{论文发表}

\begin{document}
\pagenumbering{gobble}

%***"%"后面的所有内容是注释而非代码,不会输出到最后的PDF中
%***使用本模板,只需要参照输出的PDF,在本文档的相应位置做简单替换即可
%***修改之后,输出更新后的PDF,只需要点击Overleaf中的“Recompile”按钮即可

%在大括号内填写其他信息,最多填写4个,但是如果选择不填信息,
%那么大括号必须空着不写,而不能删除大括号。
%\otherInfo后面的四个大括号里的所有信息都会在一行输出
%如果想要写两行,那就用两次这个指令(\otherInfo{}{}{}{})即可


%***********个人信息**************
\MyName{赵益民}
\sepspace
\faEnvelope \SimpleEntry{\href{mailto:ztony0712@outlook.com}{邮箱:ztony0712@outlook.com}}
\faPhone \SimpleEntry{电话:13698751959}
\faGlobe \SimpleEntry{\href{https://ztony0712.github.io/}{个人主页:https://ztony0712.github.io/}}

%************照片**************
%照片需要放到images文件夹下,名字必须是you.jpg,注意.jpg后缀(可以去resume.cls第101行处修改),如果不需要照片可以不添加此行命令
%0.15的意思是,照片的宽度是页面宽度的0.15倍,调整大小,避免遮挡文字
\yourphoto{0.14}

%***********个人总结**************
\section{个人总结}
对于将人工智能应用于各种跨学科领域十分感兴趣,尤其是生物医学信息分析和自动驾驶汽车领域。在仿真、编程、脑电图分析等方面有一定的项目经验。在新加坡国立大学高级机器人中心自动驾驶组从事研究工作,由\href{https://guppy.mpe.nus.edu.sg/~mpeangh/}{Prof. Marcelo H. Ang Jr.}指导。有较强的执行力和责任感,热衷于探索和学习。
\sepspace

%***********教育背景**************
\section{教育背景}
%***第一个大括号里的内容向左对齐,第二个大括号里的内容向右对齐
%***\textbf{}括号里的字是粗体,\textit{}括号里的字是斜体
\datedsubsection{\textbf{新加坡国立大学,新加坡}}{2023.8 - 至今}
\datedsubsection{\textit{机器人学硕士}}{GPA: 4.17/5.0}
\textit{代表课程:} 机器人视觉与AI;自主移动机器人;材料、传感器、执行器和制造技术;机器人运动学;机器人动力学与控制
\sepspace

\datedsubsection{\textbf{西南交通大学,成都}}{2019.9 - 2023.6}
\datedsubsection{\textit{计算机科学与技术学士}}{GPA: 88.6/100}
\textit{代表课程:} 机器学习;算法与数据结构;人工智能;数据挖掘;面向对象编程;软件工程;Web应用开发
\sepspace

%***********论文发表**************
\nocite{*}
\bibliographystyle{abbrv}
\bibliography{publication_list}

%***********学术经历**************
\section{学术经历}
\datedsubsection{\textbf{用于脑电图情感识别的基于相互交叉注意力机制的特征融合}}{}
\datedsubsection{\textit{中国国家自然科学基金,新加坡}}{2023.12 - 2024.3}
\begin{itemize}
    \item 提出了一种基于纯数学的相互交叉注意力机制(MCA),能更有效地融合两种特征;开发了一种独特的通道-频域-时域三维特征结构,能同时呈现频域和时域信息。
    \item 构想和设计了未来将MCA应用于Transformer以优化和提升LLM的性能。
    \item 发表于医学影像计算和计算机辅助干预 - MICCAI 2024。
\end{itemize}

\datedsubsection{\textbf{自动驾驶的运动规划仿真}}{}
\datedsubsection{\textit{硕士论文:最终项目报告,新加坡}}{2023.10 - 2024.4}
\begin{itemize}
    \item 使用nuplan-devkit提供的统一数据集和模拟器,对四种先进的规划器进行了仿真和可视化;创建了一个评估分数基准表,用于比较和分析。
    \item 基于扩散模型设计了一种新的基于学习的规划器;仿真并评估了新规划器与评估分数基准表比较。
\end{itemize}

\datedsubsection{\textbf{基于深度学习和脑电图分析的情绪判断系统}}{}
\datedsubsection{\textit{国家级大学生科研训练计划,中国}}{2021.5 - 2022.5}
\begin{itemize}
    \item 使用Python MNE包中的带通滤波器和独立成分分析(ICA)对 DEAP 数据集进行预处理。
    \item 使用连续小波分解的 db4 小波提取小波系数;从频率(128 Hz)维度(奈奎斯特速率)生成大小为64的“尺度”维度;计算每个尺度的平均能量与香农熵比(EER);选择适当的尺度范围进行计算。
    \item 通过融合双分类器,建立了一个新颖的四分类器;试验并筛选出情绪反应最突出的八个主要通道,以提高模型性能。
\end{itemize}

\datedsubsection{\textbf{基于OpenBCI-Python-Arduino的脑控多功能滚动式机器人的设计与制作}}{}
\datedsubsection{\textit{省级大学生科研训练计划,四川}}{2020.6 - 2021.5}
\begin{itemize}
    \item 利用TensorFlow中的RNN,设计并编程了一个简单的“可接受启发式”。
    \item 使用NeuroPype对脑电图数据进行预处理以去除噪声;通过通用空间模式(CSP)提取特征;使用线性判别分析(LDA)作为分类器。
    \item 用Python的PyQt设计并开发了一个新颖的“圆盘”人机交互界面。分类器的输出通过切换“圆盘”上代表不同运动状态的“档位”间接控制机器人。
\end{itemize}
\sepspace
%***********实习经历**************
\section{实习经历}
\datedsubsection{\textbf{西安臻泰智能科技有限公司}}{}
\datedsubsection{\textit{软件工程师,技术部}}{2021.7 - 2021.9}
\begin{itemize}
    \item 基于脑电图的情绪分类:创建视频实验范式;利用打标盒、脑电帽、放大器和范式建立实验平台;编写数据收集脚本。
    \item 基于脑电图的睡眠阶段监测器:实现 OSC 端口监听,以实现整体数据传输;使用 PyQt 开发睡眠阶段显示界面。
\end{itemize}

%***********社团经历**************
\section{社团经历}
\datedsubsection{\textbf{西南交通大学力学与航空航天学院}}{}
\datedsubsection{\textit{团支部书记,航模协会}}{2020.9 - 2022.6}
\begin{itemize}
    \item 竞赛组织:策划和组织第十二届和第十三届机械创新大赛;负责题目审查、材料和场地准备、宣传和结果审查。
    \item 垂直起降(VTOL)项目:启动由利兹生命基金会赞助的VTOL项目;设计结构并完成控制装置的部署。
    \item 行政工作:负责财务管理、人员招聘和文件撰写。
\end{itemize}
\sepspace

%***********获奖情况**************
\section{获奖情况}
\datedsubsection{\textbf{中美青年创客大赛} \textit{国家级二等奖、中美青年合作优胜奖}}{\textit{中华人民共和国教育部 2021.8}}
\datedsubsection{\textbf{第十七届“挑战杯”} \textit{国家级三等奖}}{\textit{共青团中央青年发展部 2021.11}}
\datedsubsection{\textbf{第十六届“挑战杯”} \textit{省级一等奖}}{\textit{共青团四川省委 2021.7}}
\datedsubsection{\textbf{美国大学生数学建模竞赛} \textit{S奖}}{\textit{美国数学及其应用联合会 2021}}
\datedsubsection{\textbf{第十三届课外科技创新实验竞赛} \textit{金奖}}{\textit{西南交通大学 2022.1}}
\datedsubsection{\textbf{第十二届课外科技创新实验竞赛} \textit{铜奖}}{\textit{西南交通大学 2021.1}}
\datedsubsection{\textbf{综合奖学金} \textit{二等奖}}{\textit{西南交通大学 2021.12}}
\datedsubsection{\textbf{优秀学生干部}}{\textit{西南交通大学 2021.12}}
\sepspace

\section{专业技能}
\datedsubsection{\textbf{编程语言}:Python, C++, MATLAB, C, HTML5, CSS3, JavaScript, Java}{}
\datedsubsection{\textbf{编程库}:PyTorch, nuplan-devkit, TensorFlow, Numpy, Flask, Django, PyQt, Qt, OpenCV}{}
\datedsubsection{\textbf{软件应用}:\LaTeX, Git, Linux, Anaconda, ROS Noetic, ROS Humble, Docker, OpenBCI, Eprime}{}
\datedsubsection{\textbf{语言}:\textit{中文(母语),英语(精通),日语(基础)}}{}
\sepspace

\end{document}