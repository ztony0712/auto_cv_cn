% Document class and font size
\documentclass[a4paper,9pt]{extarticle}

% Packages
\usepackage[utf8]{inputenc} % For input encoding
\usepackage{geometry} % For page margins
\geometry{letterpaper, margin=0.75in} % Set paper size and margins
\usepackage{titlesec} % For section title formatting
\usepackage{enumitem} % For itemized list formatting
\usepackage{hyperref} % For hyperlinks
\usepackage{tabularx}
\usepackage{fancyhdr}
\usepackage{fontawesome}

\renewcommand{\refname}{PUBLICATIONS}



% Formatting
\setlist{noitemsep} % Removes item separation
\titleformat{\section}{\large\bfseries}{\thesection}{1em}{}[\titlerule] % Section title format
\titlespacing*{\section}{0pt}{\baselineskip}{\baselineskip} % Section title spacing
%%%%%%%%%%

% Begin document
\begin{document}

% Disable page numbers
\pagestyle{fancy}
\renewcommand{\headrulewidth}{0pt}
\fancyhead{}
\fancyhead[L]{\textit{Yimin Zhao}}
\fancyhead[R]{\textit{\today}}
\thispagestyle{empty} % Remove header from the first page
\pagenumbering{gobble}

% Header
\begin{center}
\textbf{\Huge Yimin Zhao}\\ % Name

\faEnvelope \href{mailto:ztony0712@outlook.com}{ztony0712@outlook.com} | 
\faGlobe \href{https://ztony0712.github.io/}{https://ztony0712.github.io/} | 
\faHome Kunming, China % Contact info
\end{center}

\section*{PERSONAL SUMMARY}
\noindent
Interested in applying AI in various interdisciplinary fields, especially in biomedical information analysis and autonomous vehicles. Strong project experience in simulation, programming, and EEG analytics. Conducting research in Autonomous School Bus Group supervised by \href{https://guppy.mpe.nus.edu.sg/~mpeangh/}{Prof. Marcelo H. Ang Jr.} in Advanced Robotics Centre of National University of Singapore. Looking for fully funded PhD positions. \\

\noindent
\textit{Strength:} Have strong execution and sense of responsibility, passionate to explore and learn.

% Education Section
\section*{EDUCATION}
\noindent
\textbf{National University of Singapore, Singapore} \hfill Aug. 2023 | Present \\ % University name and location
\textit{Master of Science in Robotics} \hfill GPA: 4.17/5.0 \\
\textit{Supervisor:} \href{https://guppy.mpe.nus.edu.sg/~mpeangh/}{Prof. Marcelo H. Ang Jr.} \\
\textit{Representative Modules:} Robot Vision and AI; Autonomous Mobile Robotics; Materials, Sensors, Actuators and Fabrication Technologies; Robot Kinematics; Robot Dynamics and Control  \\

\noindent
\textbf{University of Leeds, Chengdu, China} \hfill Sept. 2019 | Jun. 2023 \\ % University name and location
\textit{Bachelor of Science in Computer Science (2:1)}  \hfill Average Score: 72.2/100 \\ % Degree and GPA 
\textit{Affiliation:} SWJTU-Leeds Joint School, Southwest Jiaotong University \\
\textit{Representative Modules:} Machine Learning; Algorithms and Data Structures; Artificial Intelligence; Data Mining; Object Oriented Programming; Software Engineering; Web Application Development

% PUBLICATIONS
\nocite{*}
\bibliographystyle{abbrv}
\bibliography{publication_list}

% Experience Section
\section*{ACADEMIC EXPERIENCE}

\noindent
\textbf{Feature Fusion Based on Mutual-Cross-Attention Mechanism for EEG Emotion Recognition} \\ % Project title
\textit{National Natural Science Foundation of China, Singapore} \hfill Dec. 2023 | Mar. 2024
\begin{itemize}
    \item Proposed a purely mathematical MCA which fuse two features more effectively; developed a unique Channel-Frequency-Time 3D feature structure, which presents spectral and temporal information simultaneously. 
    \item Published to Medical Image Computing and Computer Assisted Intervention -- MICCAI 2024.
    % Job responsibilities and achievements
\end{itemize}

\noindent
\textbf{Motion Planning Simulation for Autonomous Driving} \\ % Project title
\textit{Master Thesis: Final Project Report, Singapore} \hfill Oct. 2023 | Apr. 2024 % Project affiliation and location
\begin{itemize}
    \item Simulated and visualized four advanced planners using the unified dataset and simulator provided by nuplan-devkit; created an evaluation score benchmark for comparison and analysis.
    \item Designed a novel learning-based planner based on diffusion model; simulated and evaluated the new planner for comparing with the benchmark.
    % Job responsibilities and achievements
\end{itemize}

\noindent
\textbf{Emotion Judgment System Based on Deep Learning and EEG Analysis} \\ % Project title
\textit{National Student Research Training Program, China} \hfill May. 2021 | May. 2022 % Position and duration
\begin{itemize}
    \item Pre-processed DEAP dataset through band-pass filter and Independent Component Analysis (ICA) by using Python MNE package library.
    \item Extracted wavelet coefficients using the db4 wavelet of the continuous wavelet decomposition; generate 'scale' (64) dimension from frequency (128 Hz) dimension (Nyquist rate); calculated average energy to Shannon entropy ratio (EER) for each scale; selected appropriate ranges of scale to calculate.
    \item Built a novel four-classifier by merging bi-classifier; experimented and filtered the eight dominate channels with the most prominent emotional response to enhance model performance.
    % Job responsibilities and achievements
\end{itemize}

\newpage
\noindent
\textbf{Design and fabrication of a brain-controlled rolling robot based on OpenBCI-Python-Arduino} \\ % Project title
\textit{Provincial Student Research Training Program, China} \hfill Jun. 2020 | May. 2021 % Position and duration
\begin{itemize}
    \item Designed and programmed a simple 'admissible heuristic' by utilizing RNN in TensorFlow.
    \item Pre-processed EEG data using NeuroPype to remove noise; extracted feature through the Common Spatial Pattern (CSP); employed Linear Discriminant Analysis (LDA) as classifier.
    \item Designed and developed a novel 'Disk' human-computer interaction interface by PyQt of Python. The output of the classifier indirectly controlled the robot by switching 'gears' on the 'Disk' that represent different motion states.
    \item Published to International Conference on Electronics Technology (ICET), 2022.
    % Job responsibilities and achievements
\end{itemize}

% \section*{AWARDS}
% \textbf{China-US Young Maker Competition (CUYMC)} \hfill National Second Prize\\ 
% Ministry of Education of the People's Republic of China \hfill Aug. 2021\\ \\
% \textbf{17th "Challenge Up"} \hfill National Third Prize\\ 
% China Association for Science and Technology \hfill Nov. 2021\\ \\
% \textbf{16th "Challenge Up"} \hfill Provincial First Prize\\ 
% China Association for Science and Technology \hfill Jul. 2021\\ \\
% \textbf{Mathematical Contest in Modeling} \hfill Successful Participant\\ 
% Consortium for Mathematics and Its Applications \hfill 2021\\ \\
% \textbf{13th Extracurricular Scientific and Technological Innovation Experimental Competition} \hfill First Prize\\ 
% Southwest Jiaotong University \hfill Jan. 2022\\ \\
% \textbf{12th Extracurricular Scientific and Technological Innovation Experimental Competition} \hfill Third Prize\\ 
% Southwest Jiaotong University \hfill Jan. 2021\\ \\
% \textbf{Comprehensive Scholarship} \hfill Second Prize\\ 
% Southwest Jiaotong University \hfill Dec. 2021\\ \\
% \textbf{Excellent Student Cadre} \hfill Excellent\\ 
% Southwest Jiaotong University \hfill Dec. 2021\\ 

% Work experience section
\section*{WORK EXPERIENCES}
\noindent
\textbf{Xi'an ZhenTec Co., Ltd} \\ % Company
\textit{Software Engineer, Technical Department} \hfill Jul. 2021 | Sept. 2021 % Position and duration
\begin{itemize}
    \item \textit{EEG-based emotion classification:} Created video experimental paradigm; built a experimental platform utilizing marking box, EEG collector, amplifier, and paradigm; programmed data collection script.
    \item \textit{EEG-based sleep stages monitor:} Implemented OSC port listening to achieve the overall data transmission; developed sleep stages display interface using PyQt.
\end{itemize}

\section*{EXTRACURRICULAR EXPERIENCES}
\noindent
\textbf{School of Mechanical Engineering} \\ % Institution
\textit{President, Model Aircraft Association} \hfill Sept. 2020 | Jun. 2022 % Position and duration
\begin{itemize}
    \item \textit{Competition organization:} Planned and organised 12th and 13th Mechanics Innovation Competitions; responsible for the topics review, material and site preparation, publicity, and result review.
    \item \textit{Vertical take-off and landing (VTOL) project:} Initiated the VTOL project sponsored by Leeds Life Foundation; designed structure and completed the deployment of the control devices.
    \item \textit{Administrative work:} Took charge of the financial management, staff recruitment and documents writing.
\end{itemize}

\section*{HONOUR}
\begin{itemize}
    \item National Second Prize, China-US Young Maker Competition (CUYMC)
    \item Successful Participant, 2021 Mathematical Contest in Modeling
\end{itemize}

% \section*{ENGLISH \& GRE TESTS}
%     \begin{tabularx}{1\textwidth}{
%     >{\raggedright\arraybackslash}X 
%    >{\raggedright\arraybackslash}X }
%       \textbf{IELTS (Academic): 7.5} (overall score)&   \textbf{GRE General Test:}\\ \\
%     Listening: XX | Reading: XX & Quant: XXX | Verbal: XXX\\
%     Speaking: XX   | Writing: XX& Analytical writing: X\\
%     Test date: \#Month \#Year &Test date: \#Month \#Year
%     \end{tabularx}

% Skills Section
\section*{SKILLS}
\begin{itemize}
    \item \textbf{Programming:} Python, C++, MATLAB, C, HTML5, CSS3, JavaScript, Java
    \item \textbf{Libraries:} PyTorch, nuplan-devkit, TensorFlow, Numpy, Flask, Django, PyQt, Qt, OpenCV
    \item \textbf{Software:} \LaTeX, Git, Linux, Anaconda, ROS Noetic, ROS Humble, Docker, OpenBCI, Eprime
    \item \textbf{Language:} Mandarin (native), English (proficient), Japanese (basic)
\end{itemize}

\end{document}
